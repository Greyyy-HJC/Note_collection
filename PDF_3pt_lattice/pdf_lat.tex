\documentclass[prd,aps,nofootinbib,floatfix,10pt]{revtex4}
%\usepackage{amsmath,graphicx,color,epsfig}
%dsfont, mathtools
\usepackage{amsmath,graphicx,epsfig,amssymb}
\usepackage{inputenc}
\usepackage[usenames]{color}
\usepackage{ulem} %% for strike-through
\usepackage{bigstrut}
\usepackage{slashed}
\usepackage{multirow}
\usepackage{subfigure}
\renewcommand{\baselinestretch}{1.3}
\allowdisplaybreaks

\newcommand{\blue}[1]{{\color{blue} #1}}
\newcommand{\red}{\textcolor{red}}



%%%%%%%%%%%%%%%%%%%%%%%%%%%%%%%%%%%%%%%%
\begin{document}

\title{PDFs and Lattice calculation}
\author{Yu-Shan Su and Jin-Chen He}
\maketitle

\section{Basic logic}

\noindent 

We calculate 3pt correlation on lattice, then extract PDFs from 3pt correlation.


\section{Deduction}

\subsection{What is PDFs}

\noindent

sth like 

\[ \left\langle H\left(P_{z}\right)\left|\bar{\psi}(z) \gamma^{t} W(z, 0) \psi(0)\right| H\left(P_{z}\right)\right\rangle \]

*[check out Peskin 18.5.]

\subsection{Calculate PDFs through 3pt correlation}

\noindent

\[ \text{3pt} = \int d^{3} \vec{x} e^{-i \vec{p} \cdot \vec{x}} \int d^3 \vec{y}\left\langle\Omega\left|\hat{O}_{H}\left(\vec{x}, t_{s e p}\right) \hat{O}(\vec{y}, t ; z) \hat{O}_{H}^{\dagger}(0,0)\right| \Omega\right\rangle \]

in which $\hat{O}_{H}$ is projection operator, and 

\[ \hat{O}(\vec{y}, t ; z)=\bar{\psi}(z+\vec{y}, t) \gamma^{t} W(z+\vec{y}, t ; \vec{y}, t) \psi(\vec{y}, t) \]

therefore, {\color{red}ignore $t$ variables first for convenience,}

\[ \text{3pt} = \int d^{3} \vec{x} e^{-i \vec{p} \cdot \vec{x}} \int d^3 \vec{y}\left\langle\Omega\left|\hat{O}_{H}\left(\vec{x}\right) \sum_{H} \int \frac{d^3 \vec{p'}}{(2\pi)^3}  |H_{\vec{p'}}><H_{\vec{p'}}| \hat{O}(\vec{y}; z) \right. \right. \]
    
\[ \left. \left. \sum_{H'} \int \frac{d^3 \vec{p''}}{(2\pi)^3} |H'_{\vec{p''}}><H'_{\vec{p''}}| \hat{O}_{H}^{\dagger}(0)  \right| \Omega\right\rangle \]

with spatial translation operator:

\[ \hat{O}_{H}(\vec{x}) = e^{-i \hat{\vec{p}} \cdot \vec{x}} \hat{O}_{H} e^{i \hat{\vec{p}} \cdot \vec{x}} \]

so,

\[ \text{3pt} = \int d^{3} \vec{x} e^{-i \vec{p} \cdot \vec{x}} \int d^3 \vec{y}\left\langle\Omega\left|\hat{O}_{H} \cdot e^{i \hat{\vec{p}} \cdot \vec{x}} \sum_{H} \int \frac{d^3 \vec{p'}}{(2\pi)^3}  |H_{\vec{p'}}><H_{\vec{p'}}| e^{-i \hat{\vec{p}} \cdot \vec{y}} \cdot \hat{O}(0; z) \cdot e^{i \hat{\vec{p}} \cdot \vec{y}} \right. \right. \]
    
\[ \left. \left. \sum_{H'} \int \frac{d^3 \vec{p''}}{(2\pi)^3} |H'_{\vec{p''}}><H'_{\vec{p''}}| \hat{O}_{H}^{\dagger}(0)  \right| \Omega\right\rangle \]

\[= \int d^{3} \vec{x} e^{-i \vec{p} \cdot \vec{x}} \int d^3 \vec{y}\left\langle\Omega\left|\hat{O}_{H} \cdot \sum_{H} \int \frac{d^3 \vec{p'}}{(2\pi)^3} e^{i \vec{p'} \cdot \vec{x}} |H_{\vec{p'}}><H_{\vec{p'}}| e^{-i \vec{p'} \cdot \vec{y}} \cdot \hat{O}(0; z) \right. \right.\]

\[ \left. \left. \sum_{H'} \int \frac{d^3 \vec{p''}}{(2\pi)^3} e^{i \vec{p''} \cdot \vec{y}} |H'_{\vec{p''}}><H'_{\vec{p''}}| \hat{O}_{H}^{\dagger}(0)  \right| \Omega\right\rangle \]

do the integral of $x$ and $y$, then we get,

\[ \text{3pt} = \left\langle\Omega\left|\hat{O}_{H} \cdot \sum_{H} \int d^3 \vec{p'} \delta(\vec{p'} - \vec{p} ) |H_{\vec{p'}}><H_{\vec{p'}}| \hat{O}(0; z)  \sum_{H'} \int d^3 \vec{p''}  \delta(\vec{p'} - \vec{p''}) |H'_{\vec{p''}}><H'_{\vec{p''}}| \hat{O}_{H}^{\dagger}(0)  \right| \Omega\right\rangle \]

\[ = \sum_{H'} \sum_{H} <\Omega |\hat{O}_{H}  |H_{\vec{p}}><H_{\vec{p}}| \hat{O}(0; z) |H'_{\vec{p}}><H'_{\vec{p}}| \hat{O}_{H}^{\dagger}(0) | \Omega> \]


then projection operator $\hat{O}_{H}$ will select the hadron with specific quantum numbers, like for $\pi^+$, $\hat{O}_{\pi^+} = \bar{d} \gamma^5 u$.

\[ \text{3pt} = <\Omega |\hat{O}_{H} (0, t_{sep})  |H_{\vec{p}}><H_{\vec{p}}| \hat{O}(0, t; z) |H_{\vec{p}}><H_{\vec{p}}| \hat{O}_{H}^{\dagger}(0) | \Omega> \]

\[ = <\Omega |\hat{O}_{H} (0, t_{\text{sep}})  |H_{\vec{p}}> \cdot \text{PDFs} \cdot <H_{\vec{p}}| \hat{O}_{H}^{\dagger}(0) | \Omega> \]

In order to get $\text{PDFs}$, we also need 2pt correlation function:

\[ \text{2pt} = \int d^{3} \vec{x} e^{-i \vec{p} \cdot \vec{x}} <\Omega |\hat{O}_{H}\left(\vec{x}, t_{s e p}\right) \hat{O}_{H}^{\dagger}(0,0) | \Omega> \]

\[ = <\Omega |\hat{O}_{H} (0, t_{\text{sep}})  |H_{\vec{p}}>  <H_{\vec{p}}| \hat{O}_{H}^{\dagger}(0, 0) | \Omega> \]

{\color{red} Pay attention here $|H_{\vec{p}}>$ is a superposition of Hamiltonian operator instead of eigenstate. Therefore, we have the expression below (we did Wick rotation on the lattice, so $it_{M} = t_{E}$ ) }

\[ \text{3pt} = \sum_{m, n} <\Omega |\hat{O}_{H} (0, 0) e^{- \hat{H} t_{\text{sep}}}  |E_n><E_n| e^{\hat{H} t } \hat{O}(0, 0; z) e^{- \hat{H} t} |E_m><E_m| \hat{O}_{H}^{\dagger}(0) | \Omega>  \]

\[ = \sum_{m, n} e^{- E_n t_{\text{sep}}} e^{E_n t } e^{- E_m t} <\Omega |\hat{O}_{H} (0, 0)  |E_n><E_n|  \hat{O}(0, 0; z) |E_m><E_m| \hat{O}_{H}^{\dagger}(0) | \Omega> \]


same for 2pt,

\[ \text{2pt} =  \sum_{n} e^{- E_n t_{\text{sep}}} <\Omega |\hat{O}_{H} (0, 0)  |E_n>  <E_n| \hat{O}_{H}^{\dagger}(0, 0) | \Omega>  \]

For convenience, we define,

\[ z_n^{\dagger} = <E_n| \hat{O}_{H}^{\dagger}(0, 0) | \Omega> \]
\[ z_n = <\Omega |\hat{O}_{H} (0, 0)  |E_n> \]
\[ O_{n m} = <E_n|  \hat{O}(0, 0; z) |E_m> \]


preserve only lowest two energy states, we got

\[ \text{3pt} \approx z_0^2 O_{00} e^{-E_0 t_{\text{sep}}} + z_0^{\dagger} z_1 O_{01} e^{-E_0 t_{\text{sep}}} e^{- \Delta E t} + z_1^{\dagger} z_0 O_{10} e^{-E_1 t_{\text{sep}}} e^{\Delta E t} + z_1^2 O_{11} e^{- E_1 t_{\text{sep}}} \]

\[ \text{2pt} \approx z_0^2 e^{-E_0 t_{\text{sep}}} + z_1^2 e^{-E_1 t_{\text{sep}}} = z_0^2 e^{-E_0 t_{\text{sep}}} (1 + c_1 e^{-\Delta E t_{\text{sep}}}) \]

so, assume $O_{01} = O_{10}$, we have fit function

\[ \frac{\text{3pt}}{\text{2pt}} = \frac{1}{1 + c_1 e^{-\Delta E t_{\text{sep}}}} \cdot [ O_{00} + \frac{z_0^{\dagger} z_1}{z_0^2} O_{0 1} e^{-\Delta E t} + \frac{ z_1^{\dagger} z_0}{z_0^2} O_{10} e^{-\Delta E t_{\text{sep}}} e^{\Delta E t} + \frac{z_1^2}{z_0^2} O_{1 1} e^{- \Delta E t_{\text{sep}}} ] \]
{\color{red} * compared with 1, drop the $e^{- \Delta E t_{\text{sep}}}$ term in the denominator}
\[ \approx O_{00} + \frac{z_0^{\dagger} z_1}{z_0^2} O_{0 1} e^{-\Delta E t} + \frac{ z_1^{\dagger} z_0}{z_0^2} O_{10} e^{-\Delta E t_{\text{sep}}} e^{\Delta E t} + \frac{z_1^2}{z_0^2} O_{1 1} e^{- \Delta E t_{\text{sep}}} \]
\[ = O_{00} \cdot [1 + a_1 (e^{- \Delta E (t_{\text{sep}} - t)} + e^{- \Delta E t}) + a_2 e^{- \Delta E t_{\text{sep}}}]  \]

also, the FH fit function is

\[ \Sigma(t_{\text{sep}}) = \sum_{t=n}^{t_{\text{sep}} - n} \frac{\text{3pt}}{\text{2pt}} = (O_{0 0} + O_{0 0} a_2 e^{- \Delta E t_{\text{sep}}} ) \cdot (t_{\text{sep}} + 1 - 2n) + O_{0 0} a_1 e^{- \Delta E t_{\text{sep}}} \sum_{t} e^{\Delta E t} + O_{0 0} a_1 \sum_{t} e^{-\Delta E t} \]
\[ \sum_{t=n}^{t_{\text{sep}} - n} e^{\Delta E t} = e^{\Delta E n} \frac{1 - e^{\Delta E (t_{\text{sep}} + 1 - 2n)}}{1 - e^{\Delta E}} \approx \frac{e^{\Delta E n}}{1 - e^{\Delta E}} \]
\[ \Sigma(t_{\text{sep}}) = (O_{0 0} + O_{0 0} a_2 e^{- \Delta E t_{\text{sep}}} ) \cdot (t_{\text{sep}} + 1 - 2n) + \frac{O_{0 0} a_1 e^{- \Delta E (t_{\text{sep}} - n) }}{1 - e^{\Delta E}} + \frac{O_{0 0} a_1 e^{- \Delta E n}}{1 - e^{-\Delta E}} \]
\[ = (O_{0 0} + O_{0 0} a_2 e^{- \Delta E t_{\text{sep}}} ) \cdot (t_{\text{sep}} + 1 - 2n) + \frac{O_{0 0} a_1 e^{- \Delta E (t_{\text{sep}} - n + 1) }}{e^{-\Delta E} - 1} + \frac{O_{0 0} a_1 e^{- \Delta E n}}{1 - e^{-\Delta E}} \]


\[ \text{FH} = \Sigma(t_{\text{sep}} + 1) - \Sigma(t_{\text{sep}}) = O_{00} + O_{0 0} a_2 e^{- \Delta E t_{\text{sep}}} [(t_{\text{sep}} + 2 - 2n) e^{- \Delta E} - (t_{\text{sep}} + 1 - 2n) ] + O_{0 0} a_1 e^{- \Delta E (t_{\text{sep}} - n + 1) } \]
\[ \text{FH} = O_{00} \cdot [1 + a_1' e^{- \Delta E t_{\text{sep}}} + a_2' \cdot t_{\text{sep}} \cdot e^{- \Delta E t_{\text{sep}}} ]  \]

the $O_{00}$ is PDFs.





\subsection{Calculate 3pt on lattice}

For example,

\[ \text{Projection operator}\ \pi^+: \hat{O}_{\pi^+}\left(\vec{x}, t_{\text {sep }}\right)=\bar{d}\left(\vec{x}, t_{\text {sep }}\right) \gamma^{5} u\left(\vec{x}, t_{\text {sep }}\right) \]

\[ \hat{O}^{\dagger}_{\pi^+}\left(\vec{x}, t_{\text {sep }}\right)=- \bar{u}\left(\vec{x}, t_{\text {sep }}\right) \gamma^{5} d\left(\vec{x}, t_{\text {sep }}\right) \]

\[ \text{Quasi-PDF operator}\ u: \hat{O}(\vec{y}, t ; z)=\bar{u}(z+\vec{y}, t) \gamma^{t} W(z+\vec{y}, t ; \vec{y}, t) u(\vec{y}, t) \]


\[ \text{3pt} = \int d^{3} \vec{x} e^{-i \vec{p} \cdot \vec{x}} \int d^3 \vec{y}\left\langle\Omega\left|\hat{O}_{H}\left(\vec{x}, t_{s e p}\right) \hat{O}(\vec{y}, t ; z) \hat{O}_{H}^{\dagger}(0,0)\right| \Omega\right\rangle \]

\[ = \int d^{3} \vec{x} e^{-i \vec{p} \cdot \vec{x}} \int d^3 \vec{y}\left\langle\Omega\left|\bar{d}\left(\vec{x}, t_{\text {sep }}\right) \gamma^{5} u\left(\vec{x}, t_{\text {sep }}\right) \bar{u}(z+\vec{y}, t) \gamma^{t} W(z+\vec{y}, t ; \vec{y}, t) u(\vec{y}, t) \right.\right.\]

\[\left. \left. \cdot ( {\color{red}-} \bar{u}(0, 0) \gamma^{5} d(0, 0) )  \right| \Omega\right\rangle \]

Add trace, then move the $d$ at the end to the beginning. {\color{red} Notice here in the spinor space, this moving just move the column vector forward with trace, so no minus sign, while the $d$ is a dirac field, containing generation annihilation operator (or say it is Grassmann number), so this moving will contribute a minus sign. }

\[ \text{3pt} = \int d^{3} \vec{x} e^{-i \vec{p} \cdot \vec{x}} \int d^3 \vec{y}\left\langle\Omega\left| \text{tr}[ d(0, 0) \bar{d}\left(\vec{x}, t_{\text {sep}}\right) \gamma^{5} u\left(\vec{x}, t_{\text {sep}}\right)   \right.\right.\]

\[\left. \left. \bar{u}(z+\vec{y}, t) \gamma^{t} W(z+\vec{y}, t ; \vec{y}, t) u(\vec{y}, t)  \bar{u}(0, 0) \gamma^{5}  ] \right| \Omega\right\rangle \]

*[ Wick theorem Gattringer P109 (5.36), 2 $d$ fields contract, 4 $u$ have 2 kinds of contraction ]

\[ \text{3pt} = \int d^{3} \vec{x} e^{-i \vec{p} \cdot \vec{x}} \int d^3 \vec{y} \{ \]

\[ <\Omega| \text{tr}[ S_d(0,0;\vec{x},t_{\text{sep}}) \gamma^{5} S_u(\vec{x},t_{\text{sep}};z+\vec{y},t) \gamma^{t} W(z+\vec{y}, t ; \vec{y}, t) S_u(\vec{y},t;0,0) \gamma^{5}  ] |\Omega> \]

\[ - <\Omega| \text{tr}[ S_d(0,0;\vec{x},t_{\text{sep}}) \gamma^{5} S_u(\vec{x},t_{\text{sep}};0,0) \gamma^{5}] \cdot \text{tr}[ S_u(\vec{y},t;z+\vec{y},t) \gamma^{t} W(z+\vec{y}, t ; \vec{y}, t) ] |\Omega> \} \]

two terms in the integral represent two diagrams, take the first one as an example.

\[ \int d^3 \vec{y} <\Omega| \text{tr}[  \underline{  \int d^{3} \vec{x} {\color{red} e^{-i \vec{p} \cdot \vec{x}} \gamma^{5}  S_d(0,0;\vec{x},t_{\text{sep}}) \gamma^{5} } S_u(\vec{x},t_{\text{sep}};z+\vec{y},t) }  \gamma^{t} W(z+\vec{y}, t ; \vec{y}, t) S_u(\vec{y},t;0,0)  ] |\Omega> \]

in which, red part is sequential source, and the underlined part is sequential propagator.

*[ we need to avoid calculating all to all propagator, like $S_u(\vec{x},t_{\text{sep}};z+\vec{y},t)$ (x and y are both integrated) ]

\[ \int d^{3} \vec{x} e^{-i \vec{p} \cdot \vec{x}} \gamma^{5} S_{d}\left(0,0 ; \vec{x}, t_{\text {sep }}\right) \gamma^{5} S_{u}\left(\vec{x}, t_{\text {sep }} ; z+\vec{y}, t\right) \]

\[ = \gamma^5 [\int d^{3} \vec{x} S_{u}\left(z+\vec{y}, t; \vec{x}, t_{\text {sep }} \right) e^{i \vec{p} \cdot \vec{x}} \gamma^{5} S_{d}\left(\vec{x}, t_{\text {sep }}; 0,0 \right) \gamma^{5} ]^{\dagger} \gamma^5 \]

here we used 

\[ {\color{red} \gamma^5 S^{\dagger}(x;y) \gamma^5 = S(y;x) } \]

*[ Gattringer P136 (6.31)]

the $\dagger$ here acts on spinor and color indices.

So, sequential propagator:

\[ \int d^{3} \vec{x} \cdot S_{u}\left(z+\vec{y}, t; \vec{x}, t_{\text {sep }} \right) {\color{red} e^{i \vec{p} \cdot \vec{x}} \gamma^{5} S_{d}\left(\vec{x}, t_{\text {sep }}; 0,0 \right) \gamma^{5} } \]

\end{document}